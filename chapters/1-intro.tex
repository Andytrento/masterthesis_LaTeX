\section{Motivation}
 \label{sec:motivation}

In recent years, the so-called sharing economy has spread all over the world.
People share access to resources such as their homes, cars, and even personal
time with other people in a peer-to-peer manner. One of the triumphant stories
of the sharing economy in the travel and hospitality industry is Airbnb. Airbnb
is a company that pioneers in deploying this business model to let individuals'
share' extra space in their homes with those who need temporary accommodation
via the online marketplace. In 2017, Airbnb generated approximately \$93 million
in profit out of \$2.6 billion in revenue \parencite{zaleski2018}.  As of 2020,
there are 2.9 million hosts on Airbnb, with over 7 million accommodations in
100,000 cities worldwide (Airbnb 2020).

An accurate prediction of the rental price helps maximize the Airbnb host's
profits and develop their business. So far, however, little attention has been
paid to empirical Airbnb rental price prediction. This indicates a need to have
an Airbnb rental price prediction model to address the research gap.

New York City (NYC) is one of Airbnb's most popular destinations with over
50,000 listings and ranked fourth in the most popular cities for booking
experiences \parencite{airbnbfacts}. With plenty of listing and booking
activities, New York City serves as an excellent example for the study of Airbnb
pricing.

\section{Objectives}
\label{sec:objectives}

This study systematically reviews the data for Airbnb in New York City, aiming:
\begin{enumerate}
  \item To employ machine learning techniques to build a model to predict Airbnb
    listing price in New York City.
  \item To identify which features of an Airbnb listing are most important  in
predicting the price
\end{enumerate}

\section{Research Methodologies}
This thesis contains four main research methodology:
\begin{enumerate}
\item Relative to collecting data method, we scrape the information about Airbnb
  listing from the InsideAirbnb website.
\item The data preprocessing step includes cleaning, data filtering,
  normalization, transformation, etc. The product of this step is the final
  training set ready for model fitting.
\item Exploratory analysis: we use visual methods to perform initial
  investigations on data to discover patterns and trends.
\item Model fitting:
  We applied several machine learning algorithms to make Airbnb rental price
  prediction.  The comparisons between the prediction performance of those
  models are presented in chapter \ref{c:findings}.
\end{enumerate}

\section{Thesis Outline}
The overall structure of the study takes the form of five chapters. A short
description of the chapter is listed as follows:

\begin{itemize}
\item Chapter 2 - Related Works: This chapter reviews the past literature on the
pricing in the sharing economy based accommodation rentals.
\item Chapter 3 - Background: This chapter will briefly introduce machine
  learning concepts and considerations when applying machine learning to predictive
  modeling. Furthermore,  we describe the machine learning algorithms used to
  solve the research question.
\item Chapter 4 - Methodology: The third chapter is concerned with the
  methodology used for this study.
\item Chapter 5 - Experiment and Findings: In the fourth chapter, we will present the results
  we have obtained throughout our analysis.
\item Chapter 6 - Conclusions and Future Works: In the fifth chapter, we will
  answer the research question based on the findings obtained in Chapter
  \ref{c:findings}.
  Furthermore, we will state some possible future adjustments that could improve
  the results we have obtained in this thesis.
\end{itemize}

