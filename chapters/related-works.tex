
As discussed in Section ~\ref{sec:objectives}, our primary focus is to answer
the prediction question. However, besides the literature on using machine
learning techniques for price prediction, we first briefly review the related
works on the inference side.

\section{Inference}

While some research has been carried out on hotel price determinants, few
empirical investigations into the price determinants of the non-hotel
accommodation offer have been conducted.  Furthermore, most researchers
investigating the price determinant of sharing economy based accommodation have
utilized the hedonic price model.

\cite{monty2003hedonic} assessed the price determinants of bed and breakfast
amenities and confirmed the positive effects of a hot tub, a private bath, and a
larger room on room price.  In an analysis of the impact of private tourist
accommodation facilities on prices, \cite{portolan2013impact} found that the
presence of free parking places and sea view could be associated with a higher
room rate.  Both of the above studies recognized that location has a vital
influence on the price.

In recent years, there has been an increasing amount of literature on the price
determinants of Airbnb.  Several studies (\cite{gutt2015sharing};
\cite{ikkala2014defining} ) have shown that hosts who offer accommodation to
rent on Airbnb.com usually charge higher prices if their accommodation has
received high star ratings.  Using a dataset from New York City,
\cite{li2016pros,} showed that properties managed by professional hosts earn
more in daily revenue, have higher occupancy rates, and are less likely to exit
the market than properties owned by nonprofessional hosts.
\cite{kakar2016effects} measures the impact of information on hosts’ racial
background on Airbnb listing prices in San Francisco and found that Hispanic
hosts and Asian hosts, on average, have a lower list price relative to their
white counterparts.

\cite{wang2017price} analyzed the Airbnb data from 33 countries and concluded
that 24 out of 25 variables within five categories (host attributes, site, and
property attributes, amenities and services, rental rules, and the number of
online reviews and ratings) are good predictors of price.  In a similar study,
\cite{cai2019price} examines the impacts of five groups of
explanatory variables on Airbnb price in Hong Kong.


\section{Prediction}

To date, there is little published research on applying machine learning
techniques to predict the price of Airbnb listing.
\cite{tang2015neighborhood} work on the task of price prediction for San
Francisco Airbnb listings by turning the regression problem into a binary
classification problem.
\cite{li2016reasonable} has attempted to create a price prediction model for
Airbnb in different cities by using the clustering method with the distance of
the property to the city landmarks as the clustering feature.
In more recent work, \cite{kalehbasti2019airbnb}  try to develop a reliable
price prediction model using machine learning, deep learning, and natural
language processing techniques.


This study aims to contribute to this growing area of research by utilizes a
holistic approach. In particular, the contribution is as followed:
\begin{itemize}
  \item We rely on "domain knowledge" in the feature engineering process (as
    demonstrated in Chapter ~\ref{c:methodology}).
  \item We make use of visualization techniques to gain insights from data.
  \item We introduce regularization techniques to improve the disadvantage of the
  traditional ordinary least square model.
  \item We present the Boosting modeling strategy, a machine learning approach
    that performs well when there are many predictors as a baseline model.
\end{itemize}
